 \begin{refsection}
 \chapter{Methodology}

Multiple biophysical methods have been developed to investigate the molecular structure and function of membrane proteins \cite{Cournia:2015df,PebayPeyroula:2008wv}. Structural determination of membrane proteins can be achieved using x-ray crystallography, cryo-electron microscopy, or solid-state NMR \cite{Lacapere:2010fl}. With a sufficient amount of structural data, homology modelling and other bioinformatics algorithms can be utilized for membrane protein structure prediction but require careful validation \cite{Almeida:2017kq,Stansfeld:2017dk}. Protein structures are not required, but provide a concrete foundation for further biochemical and biophysical studies. Using a membrane protein activity assay, be that through electrophysiological techniques \cite{Rettinger:2016uc} or some other functional assay, it is possible to probe the molecular structure and function relationship using site-directed mutagenesis, drug-binding, or other environmental perturbations. Further quantification of membrane protein function can be determined using biophysical techniques. Circular dichroism spectroscopy can provide secondary structure information \cite{Miles:2016ks}, and fourier transform infrared spectroscopy can provide high-sensitivity measurements of conformational change \cite{Tatulian:2013hy}. Both F\"{o}rster resonance energy transfer \cite{Loura:2011ig} and and electron paramagnetic resonance \cite{Sahu:2015kd} can provide distance information between two specific sites, another effective tool to study conformational change in membrane proteins. Analytical ultracentrifugation \cite{Fleming:2016hi}, surface plasmon resonance \cite{Patching:2014bg}, atomic force microscopy \cite{Whited:2014hb}, and mass spectrometry \cite{Landreh:2014fs} serve as invaluable tools for investigating membrane protein assembly and other molecular interactions. The dynamics of proteins can be investigated at atomic resolution using serial femtosecond crystallography \cite{Zhu:2016cp} or molecular dynamics simulations \cite{Dror:2012cs,Chavent:2016cq}. Each of these approaches holds strengths and weaknesses, varying in their technical requirements, temporal, and spatial resolution. The selection of these methods is largely dependent on the timescale of the biological process of interest.

The action potential rises and falls over a few seconds. During this time, voltage-sensing domains cause sodium channels to open and ion permeation occurs. Conduction measurements for a typical voltage gated sodium channel are on the order of one sodium permeation event per microsecond \cite{Hille:2001tw}. To study the dynamics of this process, we require a method that has both atomic resolution as well as fine temporal resolution that can span up to one or more microseconds. Recent developments in high-performance computing hardware and computational algorithms have enabled the use of all-atom molecular dynamics simulations to study biological processes over a large range of timescales, from femtoseconds to microseconds (corresponding to 9 orders of magnitude). Simulations have been utilized extensively for the study of ion conduction and selectivity, and led to major discoveries in the molecular mechanism of ion permeation and selectivity \cite{Dror:2010gy}. In this chapter, we describe the methods used throughout this thesis in order to investigate the structure and function of voltage-gated sodium channels.

\section{Statistical Mechanics}

Molecular dynamics simulations are part of a group of computational methods that can be utilized to generate ensembles of atomic configurations for the calculation of macroscopic properties. We define a configuration, or microstate, as a fixed set of positions and momenta for $N$ atoms. The collection of microstates form a macrostate. For example, we might define the open state and closed state of an ion channel as two separate macrostates. The configurations accessible for a system depend on global variables held constant and define a thermodynamic ensemble. Constant variables may include the number of particles ($N$), volume ($V$), temperature ($T$), or pressure ($P$). Experimental studies are typically conducted in the canonical ensemble ($N$,$V$, and $T$ held constant) or the isothermal-isobaric ensemble ($N$,$P$, and $T$ held constant) so simulations typically employ the same approaches. The probability of finding a configuration $P_i$ is related to the total energy of the system at that microstate, $E_i$, normalized by the sum of all microstates,

\begin{equation}
\label{eq:1}
P_i = \frac{ \exp \big[ - \frac{E_{i}}{k_B T}\big]  }{  \sum_{j} \exp \big[ - \frac{E_{j}}{k_B T} \big]  }
\end{equation}

where $T$, the temperature, and $k_B$, a Boltzmann constant with units of $energy/temperature$, define a thermodynamic energy scale. This equation applies only to the canonical ensemble but a similar expression exists to describe the probability of a microstate being found in the isothermal-isobaric ensensemble. Microstates with energies higher than $k_B T$ are relatively inaccessible, with low population, and those with energies lower than $k_B T$ are more accessible and have a higher population. The total energy is a sum of the potential energy and kinetic energy at a given configuration, $E_{i}=U_{i}+K_{i}$. The denominator of equation \ref{eq:1} is the partition function. This quantity is the sum of exponential factors for each of the microstates available to the system, and specifies how microstates are partitioned into energy levels. These energy levels include the ground state energy, $E_0$, which is zero for simulations of classical systems. Here we consider a thermodynamic system with countable microstates, but for a physical system there are an infinite number of microstates and the summations in this expression and others in this section can also be written as integrals over all atom positions and momenta. The relationship between the number of accessible states and the temperature can be understood by examining the partition function in limiting cases. In the limit of low temperature, all exponential terms except for the ground-state energy term go to zero, resulting in occupancy of only a single microstate. At very high temperatures, all terms in the partition function approach $1.0$, resulting in the accessibility of all microstates. The ratio of probabilities of two microstates $i$ and $j$ correspond to their differences in their energies,

\begin{equation}
\label{eq:2}
\begin{split}
\frac{P_i}{P_j} & = \frac{ \exp \big[ - \frac{E_{i}}{k_B T}\big]  }{  \exp \big[ - \frac{E_{j}}{k_B T}\big]   } \\
                       & = \exp \Big[ - \frac{(E_{i} - E_{j})}{k_B T}\Big], \\
\end{split}
\end{equation}

which does not require the explicit calculation of a partition function. Similarly, the ratio of probabilities of two macrostates can be computed by dividing the sum of microstate probabilities within each macrostate. When the partition function is known, it can be utilized as weighting factors for the calculation of thermodynamic averages over all microstates within a macrostate. For example, the average energy $\langle E \rangle$ of a system at a given temperature is

\begin{equation} 
\label{eq:2a}
\begin{split}
 \langle E \rangle & = \sum_i E_i P_i \\
                            & = \frac{ \sum_{i} E_{i} \exp \big[ - \frac{E_{i}}{k_B T} \big]  }{  \sum_{j} \exp \big[ - \frac{E_{j}}{k_B T} \big]  }. \\
\end{split}
\end{equation}

A similar expression can be derived to calculate the thermal average of any observable. It is also possible to evaluate thermodynamic averages over a macrostate using a reduced partition function, containing a subset of microstates. Using the principles of statistical mechanics, it is possible to calculate both static equilibrium properties and dynamic or non-equilibrium properties. In the context of ion channels, examples of equilibrium properties include the ionic occupancy of binding sites and the spatial distribution of water within the pore. An example of a kinetic property is the rate of ion conduction, but this requires a time average rather than a thermodynamic average. Equilibrium properties are related to the calculation of free energy. The Helmholtz free energy (measured in the $NVT$ ensemble), $F$, and the Gibbs free energy (measured in the $NPT$ ensemble), $G$, can be computed from a partition function. In both statistical ensembles, this quantity represents the amount of work that can be obtained from the system. For a macrostate comprised of many microstates $i$,

\begin{equation}
F_i = - k_B T \ln \Big( \sum_{i} \exp \big[ - \frac{E_{i}}{k_B T} \big]  \Big).
\end{equation}

The logarithm of exponential factors is intrinsically related to free energy and can be utilized to compute differences in free energy without explicit calculation of the partition function. The difference in free energy, $\Delta F$ between macrostates $i$ and $j$ is 

\begin{equation}
\begin{split}
\Delta F & = F_i - F_j \\
              & =  -k_B T \ln \Big( \sum_{i} \exp \big[ - \frac{E_{i}}{k_B T} \big]  \Big) + k_B T \ln \Big( \sum_{j} \exp \big[ - \frac{E_{j}}{k_B T} \big]  \Big)\\
              & = -k_B T \ln \Big( \frac{\sum_{i} \exp \big[ - \frac{E_{i}}{k_B T} \big]}{\sum_{j} \exp \big[ - \frac{E_{j}}{k_B T} \big]}                         \Big) \\ 
              & = -k_B T \ln \Big( \frac{P_i}{P_j} \Big),
\end{split}
\end{equation}

where the last simplification is made using Eq. \ref{eq:1}. Free energies are utilized extensively to examine biophysical processes, but in practice, they are difficult to interpret due to their high dimensionality. In a continuous configurational integral over positions and momenta states, it is possible to project the distribution onto a lower-dimensional set of coordinates to gain greater biological insight \cite{Zuckerman:2010ue}. This is referred to as a potential of mean force (PMF, $W$) since it includes the average of all forces experienced by a particle except those on a specific reaction coordinate. The potential of mean force along a coordinate like distance, $R$, using a configurational integral over all positions and momenta ($\mathbf{r}^N$, $\mathbf{p}^N$) can be expressed as 

\begin{equation}
\label{eq:2b}
\begin{split}
 \exp \Big[ - \frac{W(R)}{k_B T} \Big] & = \frac{\int d\mathbf{r}^N d\mathbf{p}^N \delta \big( R - \hat{R}(\mathbf{r}^N) \big) \exp \big[ - \frac{E(\mathbf{r}^N, \mathbf{p}^N)}{k_B T} \big]}{\int d\mathbf{r}^N d\mathbf{p}^N \exp \big[ - \frac{E(\mathbf{r}^N, \mathbf{p}^N)}{k_B T} \big]} \\
                                                              & = P(R) \\
 \end{split}
\end{equation}

where $(\hat R)$ is a function that takes all coordinates and returns $R$, and the delta function, $\delta$, selects the positions and momenta associated with the coordinate $R$ out of the integral. $P(R)$ is a continuous probability distribution analogous to the single microstate expression given in equation \ref{eq:1}. An approximation of the PMF can be obtained, up to a constant $C$, by making a discrete estimation of $P(R)$ using a histogram of $R$ values, with the following expression,

\begin{equation}
\label{eq:3}
 W(R) = - k_{B}T \ln \big( P(R) \big) + C.
\end{equation}

In practice, the number of states and their corresponding energies cannot be simply enumerated for a complex system, so these quantities, including $P(R)$, cannot be exactly calculated. It is for this reason that the PMF is estimated relative to some reference point $R_0$, which sets the value of C. To enable numerical approximations of these values, we assume the average behaviour of an $N$-particle system can be studied by computing the time evolution of that system for a sufficiently long time and averaging over the quantity of interest. This assumption, called the ``ergodic hypothesis'', is correct when it is known that all states within a thermodynamic ensemble are accessible over long timescales. In practice, it is reasonable to assume that over finite-time simulations, all states will not be reached, but that a reasonable approximation can still be made for the calculation of observables. 

\section{Molecular Dynamics}

The objective of molecular dynamics simulations is to produce time trajectories of $N$ atoms which are guided by physical principles. With full treatment of atomic nuclei and electrons, the time evolution of $N$ atoms can be obtained by solving the time-dependent Schr\"{o}dinger equation. However, this equation can only be solved analytically for the hydrogen atom and even high-accuracy numerical methods are prohibitively expensive for multi-atom systems (with computational scaling on the order of O($M^{7}$) where $M$ is the number of basis functions)\cite{Jensen:2007wr}. As such, we use approximate models that are designed to reproduce a subset of quantum mechanical and experimental properties through parameterization of a potential energy function. Approximate models do not treat electronic degrees of freedom, and atoms are modelled as point masses with fixed charge. In addition, we make the assumption that quantum effects are negligible and that the motion of atoms can be predicted using classical physics. The time evolution of an atomic system containing $N$ interacting atoms can be described by Newton's equations. This collection of atoms forms a system of differential equations, and with a set of initial positions and momenta, it is possible to propagate the positions of atoms in type using numerical integration. Contrary to alternative sampling methods like Monte Carlo, consecutive configurations obtained from molecular dynamics are obtained using physical principles, which enables the calculation of kinetic properties.

In order to propagate the movement of atoms in time under Newton's laws, it is required to calculate the forces acting on each of $N$ atoms in the system. Forces are obtained from the gradient of the potential energy function $U$, or force field, which is comprised of both bonded and non-bonded energy terms. The CHARMM potential energy function \cite{MacKerell:1998tp}, used extensively in this work, has the form

\begin{equation}
\label{eq:4}
\begin{split}
U & = \sum_{bonds} K_b (b-b_0)^2 + \sum_{urey-bradley} K_{ub} (S-S_0)^2 + \\
    & \sum_{angles} K_{\phi} (\phi - \phi_0)^2 + \sum_{dihedrals} K_{\chi} (1+\cos(n\chi - \delta)) + \\
    & \sum_{impropers} K_{imp} (\psi - \psi_0)^2 + \sum_{lj_{i \neq j}} \epsilon_{ij} \Big[   \big(  \frac{R_{min_{ij}}}{r_{ij}} \big)^{12} - \big(  \frac{R_{min_{ij}}}{r_{ij}} \big)^{6}   \Big] + \sum_{coulomb} \frac{q_i q_j}{\epsilon_{eff} r_{ij}}, \\
\end{split}
\end{equation}

where $K_b$, $K_{ub}$, $K_{\phi}$, $K_{\chi}$, and $K_{imp}$ are empirical force constants; $b$, $S$, $\phi$, $\chi$, and $\psi$ are distance or angle variables for specific interactions, with subscript $0$ representing the equilibrium values for the terms. The first five summations indicate bonded energy terms which describe the local intramolecular interactions of biomolecules. The penultimate term defines the Lennard-Jones interactions, where $\epsilon_{ij}$ and $R_{min_{ij}}$ are the well depth and distance at the potential minimum, which includes a repulsive term at short-range and an attractive long-range term. The last contribution to the CHARMM potential energy is the Coulomb term, where $q_i$ and $q_j$ are the charges of two atoms $i$ and $j$ interacting at distance $r_{ij}$, and $\epsilon_{eff}$ is the effective dielectric constant. The last two terms of Eq. \ref{eq:4} are referred to as short-range and long-range non-bonded interactions. Since these interactions may be computed between any nearby atom of the simulation cell, not just those which are bonded, they are responsible for the most significant computational cost of molecular dynamics simulations. The functional form in Eq. \ref{eq:4} is fit using a broad range of experimental and \textit{ab initio} quantum mechanical data in order to produce a self-consistent set of parameters for proteins, water, ions, lipids, and other small molecules. The validation and improvement of force field parameterization of the CHARMM force field is an area of ongoing research \cite{Huang:2013ft,Huang:2016kb}. The calculation of forces between protein atoms and ions, critical to studies of ion channels, involves only non-bonded interactions. For example, when comparing the interaction of a sodium and potassium ions, both possessing a single positive charge, to a carboxylate oxygen atom at the same distance $r_{ij}$, the Coulombic energy term is equivalent, and the difference between these interaction are entirely determined by two parameters $R_{min_{ij}}$ and $\epsilon_{ij}$ in the calculation of the Lennard-Jones term. The optimization of these two parameters for sodium and potassium ions has been performed explicitly for the study of permeation and selectivity in ion channels \cite{Noskov:2008jp}. The potential form of equation \ref{eq:4} reveals another approximation of our molecular models; that assuming our simulations are conducted at a fixed $U$, no bonded terms can be modified, and thus no hydrogen bonds can be formed or broken. This approximation is acceptable for the study of ion channels, where no chemical bonds are broken during ion conduction. 

High-performance algorithms for molecular dynamics simulations are implemented in the software package GROMACS \cite{Abraham:2015gj,Hess:2008db,Pronk:2013ef}. This implementation includes a number of optimizations to assist in the calculation of non-bonded forces in periodic unit cells, rigid-body constraints, and support for arbitrary triclinic unit cells. GROMACS runs efficiently on GPUs as well as parallelized across hundreds of CPUs, which is essential for the simulation of membrane proteins, which may contain 100,000 or more atoms.

\section{Simulation Protocol}

A standard protocol is employed for all simulations of voltage-gated sodium channel presented in this work, and is commonly utilized within the molecular modelling and simulation community \cite{Kandt:2007wz}. This protocol proceeds though a series of steps that is performed manually, described in the following steps, but also through a similar process using the CHARMM-GUI Membrane Builder web service \cite{Wu:2014uc}: 
\begin{itemize}
\item \textbf{Membrane protein preparation}. We first construct the biological assembly of subunits that compose the entire membrane protein of interest. In the case of the voltage-gated sodium channel, a tetrameric biological assembly must be constructed, which is performed using the software CHIMERA \cite{Pettersen:2004kh}. All non-protein molecules are removed from the initial structure. Protein structures are frequently deposited into the protein databank with unresolved side chains or flexible loops. Homology modelling software is utilized to construct missing atoms and residues \cite{Fiser:2003we}. In the case of the voltage-gated sodium channel NavAb, no residues were missing within the TM region. All hydrogen atoms are added to the protein structure. N- and C-terminal groups are also constructed at the beginning and end of each peptide chain. When residues are missing from the N- and C-terminus, these residues are not explicitly modelled in our protocol. In the case of NavAb, the C-terminal domain of this protein was not resolved and these regions were not modelled. Neutral N- and C-terminal groups $-NH_2$ and $-COOH$ are added to assist in preserving the stability of the protein backbone, even though neutral termini are unphysical at physiological pH. Any single-point mutations to the structure are made using the software MODELLER at this point \cite{Fiser:2003we}. This membrane protein structure is then aligned with its principal axis along the Z-axis of the simulation cell by convention, with extracellular values positive and intracellular values negative with respect to the center of mass of the protein.
\item \textbf{Bilayer preparation}. A bilayer patch is obtained from an online repository, in a configuration after some simulation has been performed, or constructed using the CHARMM-GUI bilayer construction algorithm. In the latter case, relaxation of the simulation cell must be performed using independent molecular dynamics simulations. The size of the constructed bilayer varies depending on the size of the membrane protein construct. In the case of the voltage-gated sodium channel, we construct a rectangular bilayer patches with approximate X and Y dimensions of 83 \AA and 108 \AA for pore domain and full-length structures, respectively. It is standard practice to construct large bilayer patches with minimum periodic distance of at least twice the long-range cutoff. For some simulations of the full-length channel, we construct a bilayer patch within a truncated octahedron unit cell, which reduces the number of atoms in the box but does not necessarily decrease the distance of closest approach of the protein to its periodic image.
\item \textbf{Protein insertion into the bilayer}. Multiple algorithms exist to assist in the embedding of membrane proteins into bilayers. This process consists of removing overlapping lipid molecules, and assisting the packing of lipid molecules in the protein-lipid interface. For simulations of the voltage-gated sodium channel, we utilize $g\_membed$ \cite{Wolf:2010dr} and $alchembed$ \cite{Jefferys:2015jt} embedding protocols.
\item \textbf{Solvation and ionization}. Depending on the dimensions of the initial bilayer patch and protein embedding protocol, additional  water molecules may need be added to ensure that the protein is sufficiently solvated. There is no strong criteria for the selection of the height of the simulation cell used in our work. We do not model water molecules within the pore, but allow water molecules to enter the structure during the protein/water relaxation stage. The positions of water molecules are then exchanged for a specified concentration of NaCl, KCl, or a mixture of both. The number of ions inserted depends on the number of water molecules in the box, with a general target of 150 mM or 200 mM.
\item \textbf{Energy minimization}. After our membrane protein is inserted into a solvated lipid bilayer, steepest descent energy minimization performed with GROMACS\cite{Abraham:2015gj,Hess:2008db,Pronk:2013ef}  to remove unfavourable energy contacts that would normally cause molecular dynamics simulations to crash.
\item \textbf{System relaxation}. Initial simulations are performed with protein restraints that limit deviation away from the crystallographic structure. All simulations systems studied in this work are performed using GROMACS \cite{Abraham:2015gj,Hess:2008db,Pronk:2013ef} in the isothermal-isobaric ensemble, starting with three stages of relaxation, each lasting 10 ns. In these steps, harmonic position restraints were applied to heavy atoms, main chain backbone atoms, and then $C\alpha$ atoms, with a force constant of $1000 kJ mol^{-1} nm^{-2}$. The temperature and pressure of the simulation cell fluctuate around $300 K$ and $1 atm$, respectively. During this time, ions and water molecules flow into the channel pore and lipid molecules form closer contacts with the protein in the bilayer.
\item \textbf{Production simulations}. The final frame of system relaxation is utilized to create multiple simulation repeats. In some cases, multiple simulation repeats were conducted during the solvation and ionization stage. Each of the simulation repeats undergoes production simulations with GROMACS \cite{Abraham:2015gj,Hess:2008db,Pronk:2013ef} in the isothermal-isobaric ensemble. Data is analyzed across all simulation repeats.
\end{itemize}

\section{Umbrella Sampling}

The calculation of free energies is a foundational tool for the study of biomolecules using molecular simulation. In many cases, long timescale simulations permit the calculation of free energies along a defined order parameter, or reaction coordinate, which may assist in developing mechanistic conclusions about a biological process. In the study of ion channels, we employ this method to calculate the free energy of ion movement through the narrow confines of the channel selectivity filter using equation \ref{eq:3}. This technique works when the primary reaction coordinate and known orthogonal degrees of freedom are well-sampled over the timescale of simulations. For some channels, ions may move rapidly in and out of a well-hydrated region of the pore, resulting in reasonable estimates of free energy profiles. However, in partially or fully dehydrated region of the pore, the amount of energy required to move an ion through this constriction may require long timescale simulations, and spontaneous hydration of this region may be required before ion movement can occur. In this case, hydration and the conformation of pore lining side chains may be coupled to ion conduction, and need to be adequately sampled in order to obtain an accurate estimate of the free energy of ion conduction. The sampling of orthogonal degrees of freedom, like these, is a challenging problem for studying biological processes that occur at the temporal limits of modern simulation studies \cite{Neale:2016gj}. To assist in overcoming barriers along the reaction coordinate of interest, accelerating sampling techniques can be used. In the ``umbrella sampling'' method, an additional energy term, $U_{bias}$, is added to the total internal energy ($E=U+U_{bias}+K$) which enables the sampling of low population microstates along a reaction coordinate of interest \cite{Kastner:2011ft,Roux:1995js,Torrie:1977hs}. Enhanced sampling along a reaction coordinate $R$, is achieved by adding a harmonic biasing potential of the form

\begin{equation}
\label{eq:50}
U_{bias}(R) = \frac{1}{2} k (R-R_0)^2,
\end{equation}

where the strength of the harmonic spring is given by the force constant $k$, and the equilibrium value of the potential is placed at position $R_0$. This results in a modification to the potential energy which is similar to the addition of another bonded term in the force field. The unbiased expression for computing the probability distribution $P(R)$ is given by equation \ref{eq:2b}, and the biased expression is 

\begin{equation}
\label{eq:5}
\begin{split}
 P_{bias}(R) & = \frac{\int d\mathbf{r}^N \delta \big( R - \hat{R}(\mathbf{r}^N) \big) \exp \Big[ - \frac{U(\mathbf{r}^N) + U_{bias}(\hat{R}(\mathbf{r}^N))}{k_B T} \Big]}{\int d\mathbf{r}^N \exp \Big[ - \frac{U(\mathbf{r}^N) + U_{bias}(\hat{R}(\mathbf{r}^N))}{k_B T} \Big]} \\
                     & = \exp \Big( \frac{-U_{bias}(R)}{k_B T} \Big) \frac{\int d\mathbf{r}^N \delta \big( R - \hat{R}(\mathbf{r}^N) \big) \exp \Big[ - \frac{U(\mathbf{r}^N)}{k_B T} \Big]}{\int d\mathbf{r}^N d \exp \Big[ - \frac{U(\mathbf{r}^N) + U_{bias}(\mathbf{r}^N)}{k_B T} \Big]} \\
 \end{split}
\end{equation}

where the newly added biased potential is only dependent on $R$ in the numerator after the application of the $\delta$ function. For simplicity, here we do not include momentum in the definition of the microstate and thus we do not evaluate the kinetic energy term of the total internal energy. Combining this result with equation \ref{eq:2b} allows the unbiased probability $P(R)$ to be written as

\begin{equation}
\label{eq:5b}
\begin{split}
 P(R) & = P_{bias}(R) \times \exp \Big( \frac{U_{bias}(R)}{k_B T} \Big) \times \frac{\int d\mathbf{r}^N \exp \big[ - \frac{U(\mathbf{r}^N)}{k_B T} \big]  \exp \big[ - \frac{U_{bias}(R)}{k_B T} \big]   }{\int d\mathbf{r}^N \exp \big[ - \frac{U(\mathbf{r}^N)}{k_B T} \big]}   \\
         & = P_{bias}(R) \times \exp \Big( \frac{U_{bias}(R)}{k_B T} \Big) \times \langle \exp \big[ - \frac{U_{bias}(R)}{k_B T} \big] \rangle,  \\
 \end{split}
\end{equation}

where the thermodynamic average similar to equation \ref{eq:2a} was utilized to simplify the final term. This indicates that the unbiased probability, and thus the unbiased free energy, can be obtained by simulating the biased potential with correction terms, where $U_{bias}$ can be calculated analytically from equation \ref{eq:50}. This method can be used to combine data from multiple independent simulations with different biasing potentials to create a single PMF \cite{Hub:2010hq,Kumar:1992bv}. In the study of ion channels, this enables the sampling of ion positions along the channel pore in regions where it would be computationally prohibitive to sample using normal unbiased simulations \cite{Allen:2004vs,Allen:2006vy}.

\printbibliography[heading=subbibnumbered,title={References}]
 \end{refsection}
